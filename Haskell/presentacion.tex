\documentclass{beamer}
\usetheme{default}

\usepackage[utf8]{inputenc}
\usepackage[english,spanish]{babel}
\usepackage{amsmath,amsfonts,amssymb}
\usepackage{colortbl}
\usepackage{tikz}

\definecolor{myyellow}{rgb}{1,1,0.8}
\definecolor{mygreen}{rgb}{0.8,1,0.8}
\newcommand<>{\srccell}{\only#1{\color{blue}}}
\newcommand<>{\rescell}{\only#1{\color{red}}}
\newcommand<>{\colorhide}{\only#1{\color{lightgray}}}
%\centering\let\\\tabularnewline
\newcolumntype{M}{>{\hfill$}p{2em}<{$\hfill}}

\author{Egor Maximenko}
\institute{ESFM del IPN}
\title{División sintética}
\subtitle{también conocida como\\Algoritmo de Horner o Regla de Ruffini}
\hypersetup{pdfkeywords={división sintética,algoritmo de Horner,regla de Ruffini}}

\begin{document}
	
	\begin{frame}
	\titlepage
\end{frame}

\begin{frame}
\tableofcontents
\end{frame}

\section{Algoritmo}

\subsection{Fórmulas}
\begin{frame}{Fórmulas}
Dividir un polinomio $f(x)$ entre un binomio $(x-c)$
significa hallar un polinomio $q(x)$ y un número $r$ tales que
\[
f(x)=(x-c)q(x)+r.
\]%
\uncover<2->{Escribamos esta igualdad con más detalles:}\\\medskip
\begin{tikzpicture}
\tikzstyle{every node}=[text centered]
\uncover<3->{
\alert<4,5>{\node (f0) at (-5em,0em) {$f_0 x^n$};}
\node (f0pf1) at (-3.5em,0em) {$+$};
\alert<6,7>{\node (f1) at (-1.5em,0em) {$f_1 x^{n-1}$};}
\node (f1pf2) at (0.5em,0em) {$+$};
\alert<8,9>{\node (f2) at (2.5em,0em) {$f_2 x^{n-2}$};}
\node (f2petc) at (4.5em,0em) {$+$};
\alert<10>{\node (etctop) at (6em,0em) {$\ldots\mathstrut$};}
\node (etcpfn1) at (7em,0em) {$+$};
\alert<11,12>{\node (fn1) at (9em,0em) {$f_{n-1} x\mathstrut$};}
\node (fn1pfn) at (11em,0em) {$+$};
\alert<13,14>{\node (fn) at (12em,0em) {$f_n\mathstrut$};}
\node (eqtop) at (13em,0em) {$=$};

\node (lefteq) at (-6em,-2em) {$=$};
\node (leftbr1) at (-5.3em,-2em) {$($};
\alert<4,5,6,7,8,9,11,12>{\node (x) at (-4.8em,-2em) {$x\mathstrut$};}
\alert<6,7,8,9,11,12,13,14>{\node (xmc) at (-3.9em,-2em) {$-$};}
\alert<6,7,8,9,11,12,13,14>{\node (c) at (-2.8em,-2em) {$c\mathstrut$};}
\node (rightbr1) at (-2.4em,-2em) {$)$};
\node (leftbr2) at (-1.7em,-2em) {$($};
\alert<4,5,6,7>{\node (q0) at (0em,-2em) {$q_0 x^{n-1}$};}
\node (q0pq1) at (2em,-2em) {$+$};
\alert<6,7,8,9>{\node (q1) at (4em,-2em) {$q_1 x^{n-2}$};}
\node (q1pq2) at (6em,-2em) {$+$};
\alert<8,9>{\node (q2) at (8em,-2em) {$q_2 x^{n-3}$};}
\node (q2petc) at (10em,-2em) {$+$};
\alert<10>{\node (etcbot) at (11.2em,-2em) {$\ldots\mathstrut$};}
\node (etcqn2) at (12.2em,-2em) {$+$};
\alert<11,12>{\node (qn2) at (14.2em,-2em) {$q_{n-2} x\mathstrut$};}
\node (qn1qfn) at (16.2em,-2em) {$+$};
\alert<11,12,13,14>{\node (qn1) at (17.7em,-2em) {$q_{n-1}\mathstrut$};}
\node (rightbr2) at (19em,-2em) {$)$};
\node (pc) at (19.8em,-2em) {$+$};
\alert<13,14>{\node (r) at (20.5em,-2em) {$r$};}
\node (endpunct) at (21em,-2em) {$.\mathstrut$};
}

\visible<4,5>{
\draw[-,red] (x) .. controls +(down:4ex) and +(down:4ex) .. (q0);
}
\visible<6,7>{
\draw[-,red] (x) .. controls +(down:5ex) and +(down:5ex) .. (q1);
\draw[-,red] (c) .. controls +(down:3ex) and +(down:3ex) .. (q0);
}
\visible<8,9>{
\draw[-,red] (x) .. controls +(down:5ex) and +(down:5ex) .. (q2);
\draw[-,red] (c) .. controls +(down:3ex) and +(down:3ex) .. (q1);
}
\visible<11,12>{
\draw[-,red] (x) .. controls +(down:5ex) and +(down:5ex) .. (qn1);
\draw[-,red] (c) .. controls +(down:3ex) and +(down:3ex) .. (qn2);
}
\visible<13,14>{
\draw[-,red] (c) .. controls +(down:4ex) and +(down:4ex) .. (qn1);
}
\end{tikzpicture}\\%
\uncover<3->{Igualemos los coeficientes:}
\[
\begin{array}{rccc}
\uncover<4->{x^n\colon} &
\uncover<5->{f_0=q_0} &
\uncover<15->{\implies & q_0=f_0} \\
\uncover<6->{x^{n-1}\colon} &
\uncover<7->{f_1=q_1-c q_0} &
\uncover<15->{\implies & q_1=c q_0+f_1} \\
\uncover<8->{x^{n-2}\colon} &
\uncover<9->{f_2=q_2-c q_1} &
\uncover<15->{\implies & q_2=c q_1+f_2} \\
& \uncover<10->{\ldots} & & \\
\uncover<11->{x^1\colon} &
\uncover<12->{f_{n-1}=q_{n-1}-c q_{n-2}} &
\uncover<15->{\implies & q_{n-1}=c q_{n-2}+f_{n-1}} \\
\uncover<13->{x^0\colon} &
\uncover<14->{f_n=r-c q_{n-1}} &
\uncover<15->{\implies & r=c q_{n-1}+f_n}
\end{array}
\]
\end{frame}

\subsection{Como funciona}
\begin{frame}{Como funciona}
Usando la división sintética dividamos
el polinomio\quad $f(x)=x^3-5x^2+8$\quad entre el binomio\quad $x-2$.
\[
\tikzstyle{fcell}=[shape=rectangle,text width=1.5em,text centered,draw]
\tikzstyle{gcell}=[shape=rectangle,text width=1.5em,text centered,draw]
\begin{tikzpicture}
\uncover<2->{
\node[fcell][label=above:$f_0$] (a0) at (0,2) {$1$};
\node[fcell][label=above:$f_1$] (a1) at (1,2) {$-5$};
\node[fcell][label=above:$f_2$] (a2) at (2,2) {$0$};
\node[fcell][label=above:$f_3$] (a3) at (3,2) {$8$};

\node[gcell][label=below:$c$] (c) at (-2,0) {2};
}

\uncover<3->{
\node[gcell][label=below:$q_0$] (b0) at (0,0) {{\visible<5->{$1$}}};
\node[gcell][label=below:$q_1$] (b1) at (1,0) {{\visible<7->{$-3$}}};
\node[gcell][label=below:$q_2$] (b2) at (2,0) {{\visible<9->{$-6$}}};
\node[gcell][label=below:$r$] (r)  at (3,0) {{\visible<11->{$-4$}}};
}

\uncover<4,5>{
\draw[->] (a0) to (b0);
}

\uncover<6,7>{
\draw[->] (c) .. node[above]{$\cdot$} controls +(up:4ex) and +(up:4ex) .. (b0);
\draw[->] (b0) to node[left]{$+$} (a1);
\draw[->] (a1) to (b1);
}

\uncover<8,9>{
\draw[->] (c) .. node[above]{$\cdot$} controls +(up:4ex) and +(up:4ex) .. (b1);
\draw[->] (b1) to node[left]{$+$} (a2);
\draw[->] (a2) to (b2);
}

\uncover<10,11>{
\draw[->] (c) .. node[above]{$\cdot$} controls +(up:4ex) and +(up:4ex) .. (b2);
\draw[->] (b2) to node[left]{$+$} (a3);
\draw[->] (a3) to (r);
}
\end{tikzpicture}
\]
\begin{overlayarea}{\textwidth}{6ex}
\only<2>{Escribimos los coeficientes del polinomio dado: $f_0,f_1,f_2,f_3$.\\
Si dividimos entre $(x-c)$, entonces en la segunda fila en la izquierda escribimos $c$.}
\only<3>{Preparamos celulas vacías para el residuo $r$\\
y los coeficientes del cociente $q_0,q_1,q_2$.}
\only<4,5>{$q_0\quad :=\quad f_0\quad =\quad${\visible<5>{$1$}}}
\only<6,7>{$q_1\quad :=\quad c\cdot q_0+f_1\quad
= \quad 2\cdot 1-5\quad=\quad${\visible<7>{$-3$}}}
\only<8,9>{$q_2\quad :=\quad c\cdot q_1+f_2\quad
= \quad 2\cdot(-3)+0\quad=\quad${\visible<9>{$-6$}}}
\only<10,11>{$r\quad :=\quad c\cdot q_2+f_3\quad
= \quad 2\cdot(-6)+8\quad=\quad${\visible<11>{$-4$}}}
\only<12>{Respuesta:\quad $q(x)=x^2-3x-6$,\quad $r=-4$;\\
$f(x)=(x-2)(x^2-3x-6)-4$.}
\end{overlayarea}
\end{frame}

\subsection{En forma de tabla}

\begin{frame}{División sintética escrita en forma de tabla}
Mostremos como escribir brevemente la división del polinomio\\
$f(x)=2x^4-7x^2+6x+3$\quad entre el binomio\quad $x+1$:\\
\begin{center}
\uncover<2->{
\begin{tabular}{M|M|M|M|M|M}
& \srccell<3>2 &
\srccell<5>0 &
\srccell<7>-7 &
\srccell<9>6 &
\srccell<11>3
\\ \hline
\srccell<5,7,9,11> -1 &
\rescell<4>\srccell<5>\uncover<4->{2} &
\rescell<6>\srccell<7>\uncover<6->{-2} &
\rescell<8>\srccell<9>\uncover<8->{-5} &
\rescell<10>\srccell<11>\uncover<10->{11} &
\rescell<12>\uncover<12->{-8}
\end{tabular}
}
\end{center}
\begin{overlayarea}{\textwidth}{6ex}
\only<2>{Dibujamos una tabla de dos filas, copeamos en ella\\
los coeficientes de $f$ y el número $c$ (si dividimos entre $x-c$).}
\only<3,4>{Copeamos el mayor coeficiente:\quad{\visible<4>{$2$.}}}
\only<5,6>{Calculamos:\quad $(-1)\cdot 2+0\quad=\quad${\visible<6>{$-2$.}}}
\only<7,8>{Calculamos:\quad $(-1)\cdot(-2)+(-7)\quad=\quad${\visible<8>{$-5$.}}}
\only<9,10>{Calculamos:\quad $(-1)\cdot(-5)+6\quad=\quad${\visible<10>{$11$.}}}
\only<11,12>{Calculamos:\quad $(-1)\cdot 11+3\quad=\quad${\visible<12>{$-8$.}}}
\only<13>{Respuesta:\quad $q(x)=2x^3-2x^2-5x+11$,\quad $r=-8$.}
\end{overlayarea}
\end{frame}


\section{Aplicaciones}
\subsection{Cálculo de los valores de polinomios}

\begin{frame}{Cálculo de los valores de polinomios}
Suponemos que:
\[
f(x)=q(x)\cdot (x-c)+r.
\]

\uncover<2->{Sustituimos $x$ por $c$:}
\[
\uncover<2->{f(c)=}
\only<2>{\alert{?\phantom{.}}}\only<3->{r.}
\]

\uncover<4->{\noindent\textbf{Teorema del resto (teorema de Bézout).}\\
El valor del polinomio $f(x)$ en un punto $c$\\
es igual con el resto al dividir $f(x)$ entre $(x-c)$.
}

\bigskip
\uncover<5->{%
Calculemos el valor de $f(x)=x^3-3x^2+7x-5$ en el punto $3$:
\[
\begin{tabular}{M|MMMM}
& 1 & -3 & 7 & -5 \\ \hline
3 & 1 & 0 & 7 & \boxed{16}
\end{tabular}
\]
Respuesta: $f(3)=16$.
}
\end{frame}

\subsection{Expansión del polinomio en potencias de un binomio}

\begin{frame}{Expansión del polinomio en potencias de un binomio}
Usando el algoritmo de Horner, desarrolemos
$f(x)=x^3+3x^2-2x+4$ por las potencias de $(x+2)$.
\begin{center}
\begin{tabular}{M | M M M M}
& 1 & 3 & -2 & 4
\visible<2->{\\ -2 & 1 & 1 & -4 & \boxed{12}}
\visible<3->{\\ -2 & 1 & -1 & \boxed{-2}}
\visible<4->{\\ -2 & 1 & \boxed{-3}}
\visible<5->{\\ -2 & \boxed{1}}
\end{tabular}
\end{center}
\begin{align*}
x^3+3x^2-2x+4
\visible<2->{&= (x^2+x-4)(x+2)+12 \\}
\visible<3->{&= ((x-1)(x+2)-2)(x+2)+12 \\}
\visible<4->{&= (((1\cdot(x+2)-3)(x+2)-2)(x+2)+12 \\}
\visible<5->{&= (x+2)^3-3(x+2)^2-2(x+2)+12.}
\end{align*}
\end{frame}

\subsection{Búsqueda de ceros enteros de un polinomio}

\begin{frame}{Búsqueda de ceros enteros de un polinomio}
Usando el algoritmo de Horner, calculemos los ceros del polinomio:
\[
f(x)=x^4-2x^3+2x^2-x-6.
\]
Si un polinomio tiene coeficientes enteros,
hay que buscar sus ceros enteros entre los divisores
del coeficiente constante:
$\pm1,\pm2,\pm3,\pm6$.
\[
\begin{tabular}{MM|MMMMM}
\uncover<2->{ & & 1 & -2 &  2 & -1 & -6 \\}
\uncover<3->{  &
\colorhide<4-> 1 &
\colorhide<4-> 1 &
\colorhide<4-> -1 &
\colorhide<4-> 1 &
\colorhide<4-> 0 &
\colorhide<4-> -6 \\}
\uncover<5->{
\uncover<6->{\checkmark} & -1 &  1 & -3 &  5 & -6 & \boxed{0} \\}
\uncover<7->{ &
\colorhide<8-> -1 &
\colorhide<8-> 1 &
\colorhide<8-> -4 &
\colorhide<8-> 9 &
\colorhide<8-> -15 \\}
\uncover<9->{
\uncover<10->{\checkmark} &  2 &  1 & -1 &  3 & \boxed{0} }
\end{tabular}
\]
\[
\uncover<6->{f(x)=(x+1)(x^3-3x^2+5x-6)}
\uncover<10->{=(x+1)(x-2)(x^2-x+3).}
\]
\uncover<11->{El polinomio $x^2-x+3$ no tiene ceros enteros porque $D<0$.}
\uncover<12->{\par\noindent Respuesta: $-1$, $2$.}
\end{frame}
\end{document}
