\documentclass[10pt]{beamer}
\usetheme{metropolis}
\usepackage[utf8]{inputenc}
\usepackage[english,spanish]{babel}
\usepackage{amsmath,amsfonts,amssymb}
\usepackage{colortbl}
\usepackage{amsthm}
\usepackage{multirow}
\usepackage{xparse}
\usepackage{calc}
\usepackage{booktabs}
\usepackage{graphicx}

\newtheorem{prop}{Propocisión}
\def\realnumbers{\mathbb{R}}
\newcommand{\PhantC}{\phantom{\colon}}%
\newcommand{\PhantSQ}{\phantom{\sqrt{\hspace{0.3ex}}}}%
\newcommand{\CenterInCol}[1]{\multicolumn{1}{c}{#1}}
\newcommand{\gen}[1]{\ensuremath{\langle #1\rangle}}

% https://tex.stackexchange.com/questions/63355/wrapping-cmidrule-in-a-macro
\ExplSyntaxOn
\makeatletter
\newcommand{\CMidRule}{\noalign\bgroup\@CMidRule{}}
\NewDocumentCommand{\@CMidRule}{
	m % Material to reinsert before cmidrule.
	O{0.0ex} % #1 = left adjust
	O{0.0ex} % #1 = right adjust
	m  %       #3 = columns to span
}{
	\peek_meaning_remove_ignore_spaces:NTF \CMidRule
	{ \@CMidRule { #1 \cmidrule[\cmidrulewidth](l{#2}r{#3}){#4} } }
	{ \egroup #1 \cmidrule[\cmidrulewidth](l{#2}r{#3}){#4} }
}
\makeatother
\ExplSyntaxOff

\author{Irving Flores}
\institute{ESFM del IPN}
\title{Bases de Groebner}
\subtitle{Demostración automatica de proposiciones geométricas.}
\hypersetup{pdfkeywords={bases de Groebner, demostración automatica}}

\begin{document}

\begin{frame}
\titlepage
\end{frame}

\begin{frame}
\tableofcontents
\end{frame}

\section{Bases de Groebner}
\subsection{Motivación}

\begin{frame}{Problema principal}

	\begin{block}{Problema principal}
		Sea $I = \gen{f_1,f_2,\ldots, f_n}$ un ideal de $K[x_1,x_2,\ldots, x_m]$.
		¿Como saber si $f(x)$ pertenece al ideal?
	\end{block}

	\pause
	Es trivial cuando $m=1$, sin embargo se vuelve muy difícil en $m>1$ sin la herramienta adecuada.
\end{frame}


\begin{frame}{Algoritmo de la división}

	\begin{prop}
		Sean $f(x)$ y $g(x)\in K[x]$ entonces existen únicos $h(x)$ y $r(x)$ tales que $f(x) = g(x)h(x)+r(x)$ donde $deg(r(x)) < def(f(x))$ o $r = 0$.
	\end{prop}

	\pause
	Ejemplos:

	\begin{itemize}
		\item Para $f(x)=x^3+5x+1$ y $g(x)=x-2$ se tiene $x^3+5x+1 = (x^2+2x+9)(x-2) + 19$
		\item Para $f(x)=x^3+5x+1$ y $g(x)=x^100$ se tiene $x^3+5x+1 = (0)(x^100) + x^3+5x+1$
	\end{itemize}

\end{frame}

\begin{frame}{Algoritmo de solución en $K[x]$}
	Como $K[x]$ tiene algoritmo de la división, entonces es \emph{Dominio Euclideano} y mas particularmente \emph{DIP}.
	\begin{block}{Solución en $K[x]$}
		\begin{itemize}
			\item Calcular el único generador (g=gcd($f_1,f_2,\ldots,f_n$)).
			\item Aplicar algoritmo de la división con g.
			\item Verificar si el residuo es $0$.
		\end{itemize}
	\end{block}

	\pause
	Ejemplo:

	¿Pertenece $x^2+1$ a $\gen{3x^2, 6x^3-x}$?
	\begin{itemize}
		\item $\gcd(3x^2, 6x^3-x) = x$
		\item $x^2+1 = (x)(x)+1$
		\item Como el residuo es $1$ entonces $x^2+1$ no pertenece a $\gen{3x^2, 6x^3-x}$
	\end{itemize}
\end{frame}

\begin{frame}{El caso de $K[x_1,x_2,\ldots,x_n]$}
	\begin{block}{Algunos problemas para $K[x_1,x_2,\ldots,x_n]$}
		\begin{itemize}
			\item Se necesita definir un orden sobre los monomios.
			\item $K[x_1,x_2,\ldots,x_n]$ no es un \emph{DIP}, aunque si es Noetheriano.
			\item El algoritmo de la división clásico no sirve para el problema de la pertenencia. 
		\end{itemize}
	\end{block}
	\begin{block}{Orden monomial lexicográfico}
		Se puede definir un orden lexicografico respecto a la tupla de los exponentes de los monomios.
		
		Ejemplos:
		\begin{itemize}
			\item $x^3y^6 \leq x^7y^2$
			\item $x^5y^6z^2 \leq x^5y^6z^3$
		\end{itemize}
	\end{block}
\end{frame}

\begin{frame}{División en $K[x_1,x_2,\ldots,x_n]$}
	\begin{block}{División de polinomios, con varios \emph{divisores}}
		\begin{columns}
			\column{.5\textwidth}
				\[
					\begin{array}{rr}
					a_1\colon  & \CenterInCol{y}\\
					a_2\colon  & \CenterInCol{-1}\\
					xy + 1\PhantC & \multirow{2}*{$\overline{|xy^2 + 1}$}\\
					y + 1\PhantC &\\
					& xy^2 + y\\\cline{2-2}
					& -y + 1 \\
					& -y - 1 \\\cline{2-2}
					& 2
					\end{array}
				\]
			\pause
			\column{.5\textwidth}
				\[
					\begin{array}{rl}
					a_1\colon  & \CenterInCol{x+y}\\
					a_2\colon  & \\
					xy - 1\PhantC & \multirow{2}*{$\overline{)x^2y + xy^2 + y^2}$}\\
					y^2 - 1\PhantC &\\
					& x^2y - x\\\cline{2-2}
					& xy^2 + x + y^2 \\
					& xy^2 - y \\\cline{2-2}
					& x + y^2 + y
					\end{array}
				\]
		\end{columns}	
	\end{block}
\end{frame}

\begin{frame}{División en $K[x_1,x_2,\ldots,x_n]$}
	\scalebox{0.85}{
		$
		\begin{array}[4pt]{rll}
		a_1\colon  & \multicolumn{1}{l}{x+y}\\
		a_2\colon  & \multicolumn{1}{l}{1}           & r\\
		xy - 1\PhantC & \multirow{2}*{$\overline{)x^2y + xy^2+y^2}$} \\\cline{3-3}
		y^2 - 1\PhantC &\\
		%
		&\PhantSQ           x^2y -     x                                         \\\CMidRule[3.0ex][9.0ex]{2-2}
		&\PhantSQ \hphantom{x^2y +{}}  xy^2 +    x + y^2                         \\
		&\PhantSQ \hphantom{x^2y +{}}  xy^2 -    y                               \\\CMidRule[9.0ex][5.0ex]{2-2}
		&\PhantSQ \hphantom{x^2y +     xy^2 +{}} x +    y^2 +     y              \\\CMidRule[16.0ex][5.0ex]{2-2}
		&\PhantSQ \hphantom{x^2y +     xy^2 +    x +{}} y^2 +     y &\to x              \\
		&\PhantSQ \hphantom{x^2y +     xy^2 +    x +{}} y^2 -     1              \\\CMidRule[20.0ex][5.0ex]{2-2}
		&\PhantSQ \hphantom{x^2y +     xy^2 +    x +    y^2 +{}}  y + 1          \\\CMidRule[25.0ex][1.0ex]{2-2}
		&\PhantSQ \hphantom{x^2y +     xy^2 +    x +    y^2 + y +{}}  1 &\to x+y \\\CMidRule[25.0ex][1.0ex]{2-2}
		&\PhantSQ \hphantom{x^2y +     xy^2 +    x +    y^2 + y +{}}  0 &\to x+y+1
		\\
		% Last line above should be removed -- used for alignment purposes only.
		\end{array}
		$
	}
\end{frame}

\begin{frame}{Algoritmo de división en $K[x_1,x_2,\ldots,x_n]$ }
	\begin{prop}
		Dado un orden fijo, sean $f$ y $f_1,f_2,\ldots,f_m \in K[x_1,x_2,\ldots,x_n]$ entonces existen $a_i, r \in K[x_1,x_2,\ldots,x_n]$ tales que $f = a_1f_1+\ldots+a_mf_m + r$ donde $r=0$ o $r$ no tiene monomios que se pueden dividir por el termino líder de los $f_i$.
	\end{prop}
\end{frame}

\begin{frame}{Cambiando el orden de los \emph{divisores}}
\scalebox{0.85}{
	$
	\begin{array}[4pt]{rll}
	a_1\colon  & \multicolumn{1}{l}{x+1}\\
	a_2\colon  & \multicolumn{1}{l}{x}           & r\\
	y^2 - 1\PhantC & \multirow{2}*{$\overline{)x^2y + xy^2+y^2}$} \\\cline{3-3}
	xy - 1\PhantC &\\
	%
	&\PhantSQ           x^2y -     x                                         \\\CMidRule[3.0ex][9.0ex]{2-2}
	&\PhantSQ \hphantom{x^2y +{}}  xy^2 +    x + y^2                         \\
	&\PhantSQ \hphantom{x^2y +{}}  xy^2 -    x                               \\\CMidRule[9.0ex][5.0ex]{2-2}
	&\PhantSQ \hphantom{x^2y +     xy^2 +{}} 2x +    y^2                     \\\CMidRule[16.0ex][5.0ex]{2-2}
	&\PhantSQ \hphantom{x^2y +     xy^2 +    2x +{}} y^2  &\to 2x             \\
	&\PhantSQ \hphantom{x^2y +     xy^2 +    2x +{}} y^2 -     1              \\\CMidRule[20.0ex][5.0ex]{2-2}
	&\PhantSQ \hphantom{x^2y +     xy^2 +    2x +    y^2 +{}}  1   &\to 2x +1          \\\CMidRule[25.0ex][1.0ex]{2-2}
	&\PhantSQ \hphantom{x^2y +     xy^2 +    2x +    y^2 -{}}  0 &\to 2x+1 \\\CMidRule[25.0ex][1.0ex]{2-2}
	\\
	% Last line above should be removed -- used for alignment purposes only.
	\end{array}
	$
}
\end{frame}

\begin{frame}{Recapitulando}
	\begin{block}{Problemas con división en $K[x_1,x_2,\ldots,x_n]$}
		\begin{itemize}
			\item El residuo no es único si se cambia el orden de los \emph{divisores}.
			\item Si se cambia el orden monomial cambian los residuos.
			\item No obtener un residuo de $0$ no garantiza que el polinomio no pertenezca al ideal.
		\end{itemize}
	\end{block}
\end{frame}

\subsection{Bases de Groebner al rescate}

\begin{frame}{Que es una base de Groebner}
	\begin{block}{Definición}
		Sea $I$ un ideal en $K[x_1,x_2,\ldots,x_n]$ denotamos como $LT(I)$ como el conjunto de los términos lideres de los polinomios en $I$.
		$$LT(I) = \{cx^{\alpha} | \text{ existe } f \in I \text{ con } LT(f)=cx^{\alpha} \}$$
	\end{block}
	\pause
	\begin{block}{Proposición}
	Sea $I$ un ideal en $K[x_1,x_2,\ldots,x_n]$ entonces existen $g_1,g_2,\ldots,g_r \in I$ tales que
	$\gen{LT(I)} = \{ LT(g_1), LT(g_2), \ldots, LT(g_r)\}$
	\end{block}
	\pause
	\begin{block}{Definición}
		Dado un orden monomial fijo, un subconjunto finito $\{g_1,\ldots,g_r\}$ de $I$ se le dice \textbf{Base de Groebner} si $\gen{LT(I)} = \{ LT(g_1), LT(g_2), \ldots, LT(g_r)\}$.
	\end{block}
\end{frame}

\begin{frame}{Utilidad de la base de Groebner}
	\begin{block}{Proposición}
		Sea  $G=\{g_1,\ldots,g_r\}$ una base de Groebner del ideal $I \subset K[x_1,x_2,\ldots,x_n]$, sea $f \in K[x_1,x_2,\ldots,x_n]$ Entonces $f \in I$ si y solo si el residuo de la división de $f$ sobre $G$ es $0$
	\end{block}
	\pause
	\begin{block}{Idea de demostración}
		En cada paso de la división se tiene: $$f = ag + r$$como $f, ag \in I$ entonces $r \in I$, por lo tanto $LT(r)$ es divisible por algún $LT(g_i)$ permitiendo seguir realizando el siguiente paso hasta llegar a $0$.
	\end{block}
\end{frame}

\begin{frame}{Utilidad de la base de Groebner}
	\begin{block}{Proposición}
		Sea  $G=\{g_1,\ldots,g_r\}$ una base de Groebner del ideal $I \subset K[x_1,x_2,\ldots,x_n]$, sea $f \in K[x_1,x_2,\ldots,x_n]$ Entonces $f \in I$ si y solo si el residuo de la división de $f$ sobre $G$ es $0$
	\end{block}
	\pause
	\begin{block}{Idea de demostración}
		En cada paso de la división se tiene: $$f = ag + r$$como $f, ag \in I$ entonces $r \in I$, por lo tanto $LT(r)$ es divisible por algún $LT(g_i)$ permitiendo seguir realizando el siguiente paso hasta llegar a $0$.
	\end{block}
\end{frame}

\begin{frame}{Algoritmo de Buchberger}
	\begin{itemize}
		\item En 1949 Groebner sugirió esto acerca de las bases.
		\item En 1965 Buchberger alumno de Groebner descubrió el algoritmo para encontrar las bases.
		\item Le dio el honor a su profesor llamando con su nombre a las bases.
		\item El algoritmo es bastante sencillo de implementar en una computadora.
	\end{itemize}

	\pause
	Ejemplos:
	
	si $I = \gen{x^2 - 2y^2, xy - 3}$ entonces $G=\gen{x - \frac{2}{3}y^3, y^4 - \frac{9}{2}}$
	
	si $I=\gen{x^3-y^6, x^{15}-y^{10}}$ entonces $G=\gen{x^{15} - x^3y^4, y^6 - x^3}$
\end{frame}

\section{Demostración automática de proposiciones geométricas}
\subsection{Traducción de problemas geométricos}

\begin{frame}{Traduciendo proposiciones geométricas a polinomios}
	Considerando el plano cartesiano, se tiene que es posible representar las siguientes proposiciones como:
	\begin{itemize}
		\item {El punto $(x,y)$ es igual a $(a,b)$}
			\begin{itemize}
				\item $x-a$
				\item $y-b$
			\end{itemize}
		\pause
		\item La recta que une $(a_1,a_2)$ con $(b_1,b_2)$ es paralela a la recta que une $(c_1,c_2)$ con $(d_1,d_2)$
			\begin{itemize}
				\item $(b_2-a_2)(d_1-c_1)-(d_2-c_2)(b_1-a_1)$
			\end{itemize}
		\pause
		\item La recta que une $(a_1,a_2)$ con $(b_1,b_2)$ es perpendicular a la recta que une $(c_1,c_2)$ con $(d_1,d_2)$
			\begin{itemize}
				\item $(b_2-a_2)(d_2-c_2)-(d_1-c_1)(b_1-a_1)$
			\end{itemize}
		\pause
		\item El punto $(x,y)$ se encuentra en la recta que une $(a_1,a_2)$ con $(b_1,b_2)$.
			\begin{itemize}
				\item $(x-a_1)(b_1-a_1)-(b_2-a_2)(y-a_2)$
			\end{itemize}
	\end{itemize}
\end{frame}
\begin{frame}{Conexión con bases de Groebner}
	\begin{block}{Proposición}
		Sean $h_1,h_2, \ldots,h_l,c \in K[x_1,x_2,\ldots,x_n]$ polinomios donde cada $h_i$ representan hipótesis geométricas y $c$ una conclusión, entonces $h_1,h_2, \ldots,h_l$ implican $c$ si y solo si $c \in \gen{h_1,h_2,\ldots,h_l}$.
	\end{block}
	\begin{block}{Idea de demostración}
		Si $c \in \gen{h_1,h_2,\ldots,h_l}$ entonces, algún punto $p$ en $K^n$ que satisfaga que $g_i(p)=0$ $\forall$ $i=1,\ldots,l$ entonces también cumple que $c(p)=0$.
	\end{block}
\end{frame}
\subsection{Ejemplos de proposiciones}
\begin{frame}{Ejemplo muy sencillo}

	\begin{block}{Proposición}
		El punto $(2,2)$ pertenece a la recta que une el centro $(0,0)$ con el punto $(6,6)$.
	\end{block}

	Traduciendo, tenemos que:

	\begin{itemize} 
		\item $h_1(x,y) = 6x-6y$
		\item $h_2(x,y) = x-2$
		\item $c(x,y) = y-2$
	\end{itemize}

	Luego $c \in \gen{h_1,h_2}$
\end{frame}

\begin{frame}{Dos lineas paralelas no se cruzan}

\begin{block}{Proposición}
	El punto $(2,2)$ pertenece a la recta que une el centro $(0,0)$ con el punto $(6,6)$.
\end{block}

Traduciendo, tenemos que:

\begin{itemize} 
	\item $h_1(x,y) = 6x-6y$
	\item $h_2(x,y) = x-2$
	\item $c(x,y) = y-2$
\end{itemize}

Luego $c \in \gen{h_1,h_2}$
\end{frame}
\end{document}

