\documentclass{beamer}
\usetheme{metropolis}
\usepackage[utf8]{inputenc}
\usepackage[english,spanish]{babel}
\usepackage{amsmath,amsfonts,amssymb}
\usepackage{colortbl}
\usepackage{amsthm}
\newtheorem{prop}{Propocisión}
\def\realnumbers{\mathbb{R}}

\author{Irving Flores}
\institute{ESFM del IPN}
\title{Bases de Groebner}
\subtitle{Demostración automatica de proposiciones geométricas.}
\hypersetup{pdfkeywords={bases de Groebner, demostración automatica}}

\begin{document}

\begin{frame}
\titlepage
\end{frame}

\begin{frame}
\tableofcontents
\end{frame}

\section{Bases de Groebner}

\subsection{Motivación}

\begin{frame}{Problema principal}

	\begin{block}{Problema principal}
		Sea $I = <f_1,f_2,\ldots, f_n>$ un ideal de $K[x_1,x_2,\ldots, x_m]$.
		¿Como saber si $f(x)$ pertenece al ideal?
	\end{block}

	\pause
	Es trivial cuando $m=1$, sin embargo se vuelve muy difícil en $m>1$ sin la herramienta adecuada.
\end{frame}


\begin{frame}{Algoritmo de la división}

	\begin{prop}
		Sean $f(x)$ y $g(x)\in K[x]$ entonces existen únicos $h(x)$ y $r(x)$ tales que $f(x) = g(x)h(x)+r(x)$ donde $deg(r(x)) < def(f(x))$ o $r = 0$.
	\end{prop}

	\pause
	Ejemplos:

	\begin{itemize}
		\item Para $f(x)=x^3+5x+1$ y $g(x)=x-2$ se tiene $x^3+5x+1 = (x^2+2x+9)(x-2) + 19$
		\item Para $f(x)=x^3+5x+1$ y $g(x)=x^100$ se tiene $x^3+5x+1 = (0)(x^100) + x^3+5x+1$
	\end{itemize}

\end{frame}

\begin{frame}{Algoritmo de solución en $K[x]$}
	Como $K[x]$ tiene algoritmo de la división, entonces es \emph{Dominio Euclideano} y mas particularmente \emph{DIP}.
	\begin{block}{Solución en $K[x]$}
		\begin{itemize}
			\item Calcular el único generador (g=gcd($f_1,f_2,\ldots,f_n$)).
			\item Aplicar algoritmo de la división con g.
			\item Verificar si el residuo es $0$.
		\end{itemize}
	\end{block}

	\pause
	Ejemplo:

	¿Pertenece $x^2+1$ a $<3x^2, 6x^3-x>$?
	\begin{itemize}
		\item $\gcd(3x^2, 6x^3-x) = x$
		\item $x^2+1 = (x)(x)+1$
		\item Como el residuo es $1$ entonces $x^2+1$ no pertenece a $<3x^2, 6x^3-x>$
	\end{itemize}
\end{frame}
\end{document}
